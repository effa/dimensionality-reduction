
% Default to the notebook output style

    


% Inherit from the specified cell style.




    
\documentclass[11pt]{article}

    
    
    \usepackage[T1]{fontenc}
    % Nicer default font (+ math font) than Computer Modern for most use cases
    \usepackage{mathpazo}

    % Basic figure setup, for now with no caption control since it's done
    % automatically by Pandoc (which extracts ![](path) syntax from Markdown).
    \usepackage{graphicx}
    % We will generate all images so they have a width \maxwidth. This means
    % that they will get their normal width if they fit onto the page, but
    % are scaled down if they would overflow the margins.
    \makeatletter
    \def\maxwidth{\ifdim\Gin@nat@width>\linewidth\linewidth
    \else\Gin@nat@width\fi}
    \makeatother
    \let\Oldincludegraphics\includegraphics
    % Set max figure width to be 80% of text width, for now hardcoded.
    \renewcommand{\includegraphics}[1]{\Oldincludegraphics[width=.8\maxwidth]{#1}}
    % Ensure that by default, figures have no caption (until we provide a
    % proper Figure object with a Caption API and a way to capture that
    % in the conversion process - todo).
    \usepackage{caption}
    \DeclareCaptionLabelFormat{nolabel}{}
    \captionsetup{labelformat=nolabel}

    \usepackage{adjustbox} % Used to constrain images to a maximum size 
    \usepackage{xcolor} % Allow colors to be defined
    \usepackage{enumerate} % Needed for markdown enumerations to work
    \usepackage{geometry} % Used to adjust the document margins
    \usepackage{amsmath} % Equations
    \usepackage{amssymb} % Equations
    \usepackage{textcomp} % defines textquotesingle
    % Hack from http://tex.stackexchange.com/a/47451/13684:
    \AtBeginDocument{%
        \def\PYZsq{\textquotesingle}% Upright quotes in Pygmentized code
    }
    \usepackage{upquote} % Upright quotes for verbatim code
    \usepackage{eurosym} % defines \euro
    \usepackage[mathletters]{ucs} % Extended unicode (utf-8) support
    \usepackage[utf8x]{inputenc} % Allow utf-8 characters in the tex document
    \usepackage{fancyvrb} % verbatim replacement that allows latex
    \usepackage{grffile} % extends the file name processing of package graphics 
                         % to support a larger range 
    % The hyperref package gives us a pdf with properly built
    % internal navigation ('pdf bookmarks' for the table of contents,
    % internal cross-reference links, web links for URLs, etc.)
    \usepackage{hyperref}
    \usepackage{longtable} % longtable support required by pandoc >1.10
    \usepackage{booktabs}  % table support for pandoc > 1.12.2
    \usepackage[inline]{enumitem} % IRkernel/repr support (it uses the enumerate* environment)
    \usepackage[normalem]{ulem} % ulem is needed to support strikethroughs (\sout)
                                % normalem makes italics be italics, not underlines
    

    
    
    % Colors for the hyperref package
    \definecolor{urlcolor}{rgb}{0,.145,.698}
    \definecolor{linkcolor}{rgb}{.71,0.21,0.01}
    \definecolor{citecolor}{rgb}{.12,.54,.11}

    % ANSI colors
    \definecolor{ansi-black}{HTML}{3E424D}
    \definecolor{ansi-black-intense}{HTML}{282C36}
    \definecolor{ansi-red}{HTML}{E75C58}
    \definecolor{ansi-red-intense}{HTML}{B22B31}
    \definecolor{ansi-green}{HTML}{00A250}
    \definecolor{ansi-green-intense}{HTML}{007427}
    \definecolor{ansi-yellow}{HTML}{DDB62B}
    \definecolor{ansi-yellow-intense}{HTML}{B27D12}
    \definecolor{ansi-blue}{HTML}{208FFB}
    \definecolor{ansi-blue-intense}{HTML}{0065CA}
    \definecolor{ansi-magenta}{HTML}{D160C4}
    \definecolor{ansi-magenta-intense}{HTML}{A03196}
    \definecolor{ansi-cyan}{HTML}{60C6C8}
    \definecolor{ansi-cyan-intense}{HTML}{258F8F}
    \definecolor{ansi-white}{HTML}{C5C1B4}
    \definecolor{ansi-white-intense}{HTML}{A1A6B2}

    % commands and environments needed by pandoc snippets
    % extracted from the output of `pandoc -s`
    \providecommand{\tightlist}{%
      \setlength{\itemsep}{0pt}\setlength{\parskip}{0pt}}
    \DefineVerbatimEnvironment{Highlighting}{Verbatim}{commandchars=\\\{\}}
    % Add ',fontsize=\small' for more characters per line
    \newenvironment{Shaded}{}{}
    \newcommand{\KeywordTok}[1]{\textcolor[rgb]{0.00,0.44,0.13}{\textbf{{#1}}}}
    \newcommand{\DataTypeTok}[1]{\textcolor[rgb]{0.56,0.13,0.00}{{#1}}}
    \newcommand{\DecValTok}[1]{\textcolor[rgb]{0.25,0.63,0.44}{{#1}}}
    \newcommand{\BaseNTok}[1]{\textcolor[rgb]{0.25,0.63,0.44}{{#1}}}
    \newcommand{\FloatTok}[1]{\textcolor[rgb]{0.25,0.63,0.44}{{#1}}}
    \newcommand{\CharTok}[1]{\textcolor[rgb]{0.25,0.44,0.63}{{#1}}}
    \newcommand{\StringTok}[1]{\textcolor[rgb]{0.25,0.44,0.63}{{#1}}}
    \newcommand{\CommentTok}[1]{\textcolor[rgb]{0.38,0.63,0.69}{\textit{{#1}}}}
    \newcommand{\OtherTok}[1]{\textcolor[rgb]{0.00,0.44,0.13}{{#1}}}
    \newcommand{\AlertTok}[1]{\textcolor[rgb]{1.00,0.00,0.00}{\textbf{{#1}}}}
    \newcommand{\FunctionTok}[1]{\textcolor[rgb]{0.02,0.16,0.49}{{#1}}}
    \newcommand{\RegionMarkerTok}[1]{{#1}}
    \newcommand{\ErrorTok}[1]{\textcolor[rgb]{1.00,0.00,0.00}{\textbf{{#1}}}}
    \newcommand{\NormalTok}[1]{{#1}}
    
    % Additional commands for more recent versions of Pandoc
    \newcommand{\ConstantTok}[1]{\textcolor[rgb]{0.53,0.00,0.00}{{#1}}}
    \newcommand{\SpecialCharTok}[1]{\textcolor[rgb]{0.25,0.44,0.63}{{#1}}}
    \newcommand{\VerbatimStringTok}[1]{\textcolor[rgb]{0.25,0.44,0.63}{{#1}}}
    \newcommand{\SpecialStringTok}[1]{\textcolor[rgb]{0.73,0.40,0.53}{{#1}}}
    \newcommand{\ImportTok}[1]{{#1}}
    \newcommand{\DocumentationTok}[1]{\textcolor[rgb]{0.73,0.13,0.13}{\textit{{#1}}}}
    \newcommand{\AnnotationTok}[1]{\textcolor[rgb]{0.38,0.63,0.69}{\textbf{\textit{{#1}}}}}
    \newcommand{\CommentVarTok}[1]{\textcolor[rgb]{0.38,0.63,0.69}{\textbf{\textit{{#1}}}}}
    \newcommand{\VariableTok}[1]{\textcolor[rgb]{0.10,0.09,0.49}{{#1}}}
    \newcommand{\ControlFlowTok}[1]{\textcolor[rgb]{0.00,0.44,0.13}{\textbf{{#1}}}}
    \newcommand{\OperatorTok}[1]{\textcolor[rgb]{0.40,0.40,0.40}{{#1}}}
    \newcommand{\BuiltInTok}[1]{{#1}}
    \newcommand{\ExtensionTok}[1]{{#1}}
    \newcommand{\PreprocessorTok}[1]{\textcolor[rgb]{0.74,0.48,0.00}{{#1}}}
    \newcommand{\AttributeTok}[1]{\textcolor[rgb]{0.49,0.56,0.16}{{#1}}}
    \newcommand{\InformationTok}[1]{\textcolor[rgb]{0.38,0.63,0.69}{\textbf{\textit{{#1}}}}}
    \newcommand{\WarningTok}[1]{\textcolor[rgb]{0.38,0.63,0.69}{\textbf{\textit{{#1}}}}}
    
    
    % Define a nice break command that doesn't care if a line doesn't already
    % exist.
    \def\br{\hspace*{\fill} \\* }
    % Math Jax compatability definitions
    \def\gt{>}
    \def\lt{<}
    % Document parameters
    \title{feature-selection}
    
    
    

    % Pygments definitions
    
\makeatletter
\def\PY@reset{\let\PY@it=\relax \let\PY@bf=\relax%
    \let\PY@ul=\relax \let\PY@tc=\relax%
    \let\PY@bc=\relax \let\PY@ff=\relax}
\def\PY@tok#1{\csname PY@tok@#1\endcsname}
\def\PY@toks#1+{\ifx\relax#1\empty\else%
    \PY@tok{#1}\expandafter\PY@toks\fi}
\def\PY@do#1{\PY@bc{\PY@tc{\PY@ul{%
    \PY@it{\PY@bf{\PY@ff{#1}}}}}}}
\def\PY#1#2{\PY@reset\PY@toks#1+\relax+\PY@do{#2}}

\expandafter\def\csname PY@tok@nd\endcsname{\def\PY@tc##1{\textcolor[rgb]{0.67,0.13,1.00}{##1}}}
\expandafter\def\csname PY@tok@go\endcsname{\def\PY@tc##1{\textcolor[rgb]{0.53,0.53,0.53}{##1}}}
\expandafter\def\csname PY@tok@mh\endcsname{\def\PY@tc##1{\textcolor[rgb]{0.40,0.40,0.40}{##1}}}
\expandafter\def\csname PY@tok@kp\endcsname{\def\PY@tc##1{\textcolor[rgb]{0.00,0.50,0.00}{##1}}}
\expandafter\def\csname PY@tok@si\endcsname{\let\PY@bf=\textbf\def\PY@tc##1{\textcolor[rgb]{0.73,0.40,0.53}{##1}}}
\expandafter\def\csname PY@tok@gr\endcsname{\def\PY@tc##1{\textcolor[rgb]{1.00,0.00,0.00}{##1}}}
\expandafter\def\csname PY@tok@nn\endcsname{\let\PY@bf=\textbf\def\PY@tc##1{\textcolor[rgb]{0.00,0.00,1.00}{##1}}}
\expandafter\def\csname PY@tok@s\endcsname{\def\PY@tc##1{\textcolor[rgb]{0.73,0.13,0.13}{##1}}}
\expandafter\def\csname PY@tok@gs\endcsname{\let\PY@bf=\textbf}
\expandafter\def\csname PY@tok@k\endcsname{\let\PY@bf=\textbf\def\PY@tc##1{\textcolor[rgb]{0.00,0.50,0.00}{##1}}}
\expandafter\def\csname PY@tok@sd\endcsname{\let\PY@it=\textit\def\PY@tc##1{\textcolor[rgb]{0.73,0.13,0.13}{##1}}}
\expandafter\def\csname PY@tok@ne\endcsname{\let\PY@bf=\textbf\def\PY@tc##1{\textcolor[rgb]{0.82,0.25,0.23}{##1}}}
\expandafter\def\csname PY@tok@vc\endcsname{\def\PY@tc##1{\textcolor[rgb]{0.10,0.09,0.49}{##1}}}
\expandafter\def\csname PY@tok@sr\endcsname{\def\PY@tc##1{\textcolor[rgb]{0.73,0.40,0.53}{##1}}}
\expandafter\def\csname PY@tok@ch\endcsname{\let\PY@it=\textit\def\PY@tc##1{\textcolor[rgb]{0.25,0.50,0.50}{##1}}}
\expandafter\def\csname PY@tok@sc\endcsname{\def\PY@tc##1{\textcolor[rgb]{0.73,0.13,0.13}{##1}}}
\expandafter\def\csname PY@tok@vi\endcsname{\def\PY@tc##1{\textcolor[rgb]{0.10,0.09,0.49}{##1}}}
\expandafter\def\csname PY@tok@nb\endcsname{\def\PY@tc##1{\textcolor[rgb]{0.00,0.50,0.00}{##1}}}
\expandafter\def\csname PY@tok@gp\endcsname{\let\PY@bf=\textbf\def\PY@tc##1{\textcolor[rgb]{0.00,0.00,0.50}{##1}}}
\expandafter\def\csname PY@tok@sh\endcsname{\def\PY@tc##1{\textcolor[rgb]{0.73,0.13,0.13}{##1}}}
\expandafter\def\csname PY@tok@na\endcsname{\def\PY@tc##1{\textcolor[rgb]{0.49,0.56,0.16}{##1}}}
\expandafter\def\csname PY@tok@cm\endcsname{\let\PY@it=\textit\def\PY@tc##1{\textcolor[rgb]{0.25,0.50,0.50}{##1}}}
\expandafter\def\csname PY@tok@c\endcsname{\let\PY@it=\textit\def\PY@tc##1{\textcolor[rgb]{0.25,0.50,0.50}{##1}}}
\expandafter\def\csname PY@tok@kr\endcsname{\let\PY@bf=\textbf\def\PY@tc##1{\textcolor[rgb]{0.00,0.50,0.00}{##1}}}
\expandafter\def\csname PY@tok@ow\endcsname{\let\PY@bf=\textbf\def\PY@tc##1{\textcolor[rgb]{0.67,0.13,1.00}{##1}}}
\expandafter\def\csname PY@tok@se\endcsname{\let\PY@bf=\textbf\def\PY@tc##1{\textcolor[rgb]{0.73,0.40,0.13}{##1}}}
\expandafter\def\csname PY@tok@mf\endcsname{\def\PY@tc##1{\textcolor[rgb]{0.40,0.40,0.40}{##1}}}
\expandafter\def\csname PY@tok@dl\endcsname{\def\PY@tc##1{\textcolor[rgb]{0.73,0.13,0.13}{##1}}}
\expandafter\def\csname PY@tok@kd\endcsname{\let\PY@bf=\textbf\def\PY@tc##1{\textcolor[rgb]{0.00,0.50,0.00}{##1}}}
\expandafter\def\csname PY@tok@mo\endcsname{\def\PY@tc##1{\textcolor[rgb]{0.40,0.40,0.40}{##1}}}
\expandafter\def\csname PY@tok@sx\endcsname{\def\PY@tc##1{\textcolor[rgb]{0.00,0.50,0.00}{##1}}}
\expandafter\def\csname PY@tok@gh\endcsname{\let\PY@bf=\textbf\def\PY@tc##1{\textcolor[rgb]{0.00,0.00,0.50}{##1}}}
\expandafter\def\csname PY@tok@s2\endcsname{\def\PY@tc##1{\textcolor[rgb]{0.73,0.13,0.13}{##1}}}
\expandafter\def\csname PY@tok@il\endcsname{\def\PY@tc##1{\textcolor[rgb]{0.40,0.40,0.40}{##1}}}
\expandafter\def\csname PY@tok@sa\endcsname{\def\PY@tc##1{\textcolor[rgb]{0.73,0.13,0.13}{##1}}}
\expandafter\def\csname PY@tok@nv\endcsname{\def\PY@tc##1{\textcolor[rgb]{0.10,0.09,0.49}{##1}}}
\expandafter\def\csname PY@tok@kt\endcsname{\def\PY@tc##1{\textcolor[rgb]{0.69,0.00,0.25}{##1}}}
\expandafter\def\csname PY@tok@nc\endcsname{\let\PY@bf=\textbf\def\PY@tc##1{\textcolor[rgb]{0.00,0.00,1.00}{##1}}}
\expandafter\def\csname PY@tok@nt\endcsname{\let\PY@bf=\textbf\def\PY@tc##1{\textcolor[rgb]{0.00,0.50,0.00}{##1}}}
\expandafter\def\csname PY@tok@vm\endcsname{\def\PY@tc##1{\textcolor[rgb]{0.10,0.09,0.49}{##1}}}
\expandafter\def\csname PY@tok@w\endcsname{\def\PY@tc##1{\textcolor[rgb]{0.73,0.73,0.73}{##1}}}
\expandafter\def\csname PY@tok@fm\endcsname{\def\PY@tc##1{\textcolor[rgb]{0.00,0.00,1.00}{##1}}}
\expandafter\def\csname PY@tok@ge\endcsname{\let\PY@it=\textit}
\expandafter\def\csname PY@tok@bp\endcsname{\def\PY@tc##1{\textcolor[rgb]{0.00,0.50,0.00}{##1}}}
\expandafter\def\csname PY@tok@gd\endcsname{\def\PY@tc##1{\textcolor[rgb]{0.63,0.00,0.00}{##1}}}
\expandafter\def\csname PY@tok@ss\endcsname{\def\PY@tc##1{\textcolor[rgb]{0.10,0.09,0.49}{##1}}}
\expandafter\def\csname PY@tok@nl\endcsname{\def\PY@tc##1{\textcolor[rgb]{0.63,0.63,0.00}{##1}}}
\expandafter\def\csname PY@tok@o\endcsname{\def\PY@tc##1{\textcolor[rgb]{0.40,0.40,0.40}{##1}}}
\expandafter\def\csname PY@tok@cs\endcsname{\let\PY@it=\textit\def\PY@tc##1{\textcolor[rgb]{0.25,0.50,0.50}{##1}}}
\expandafter\def\csname PY@tok@s1\endcsname{\def\PY@tc##1{\textcolor[rgb]{0.73,0.13,0.13}{##1}}}
\expandafter\def\csname PY@tok@vg\endcsname{\def\PY@tc##1{\textcolor[rgb]{0.10,0.09,0.49}{##1}}}
\expandafter\def\csname PY@tok@gt\endcsname{\def\PY@tc##1{\textcolor[rgb]{0.00,0.27,0.87}{##1}}}
\expandafter\def\csname PY@tok@cp\endcsname{\def\PY@tc##1{\textcolor[rgb]{0.74,0.48,0.00}{##1}}}
\expandafter\def\csname PY@tok@gu\endcsname{\let\PY@bf=\textbf\def\PY@tc##1{\textcolor[rgb]{0.50,0.00,0.50}{##1}}}
\expandafter\def\csname PY@tok@kn\endcsname{\let\PY@bf=\textbf\def\PY@tc##1{\textcolor[rgb]{0.00,0.50,0.00}{##1}}}
\expandafter\def\csname PY@tok@mi\endcsname{\def\PY@tc##1{\textcolor[rgb]{0.40,0.40,0.40}{##1}}}
\expandafter\def\csname PY@tok@cpf\endcsname{\let\PY@it=\textit\def\PY@tc##1{\textcolor[rgb]{0.25,0.50,0.50}{##1}}}
\expandafter\def\csname PY@tok@sb\endcsname{\def\PY@tc##1{\textcolor[rgb]{0.73,0.13,0.13}{##1}}}
\expandafter\def\csname PY@tok@ni\endcsname{\let\PY@bf=\textbf\def\PY@tc##1{\textcolor[rgb]{0.60,0.60,0.60}{##1}}}
\expandafter\def\csname PY@tok@mb\endcsname{\def\PY@tc##1{\textcolor[rgb]{0.40,0.40,0.40}{##1}}}
\expandafter\def\csname PY@tok@err\endcsname{\def\PY@bc##1{\setlength{\fboxsep}{0pt}\fcolorbox[rgb]{1.00,0.00,0.00}{1,1,1}{\strut ##1}}}
\expandafter\def\csname PY@tok@kc\endcsname{\let\PY@bf=\textbf\def\PY@tc##1{\textcolor[rgb]{0.00,0.50,0.00}{##1}}}
\expandafter\def\csname PY@tok@m\endcsname{\def\PY@tc##1{\textcolor[rgb]{0.40,0.40,0.40}{##1}}}
\expandafter\def\csname PY@tok@nf\endcsname{\def\PY@tc##1{\textcolor[rgb]{0.00,0.00,1.00}{##1}}}
\expandafter\def\csname PY@tok@gi\endcsname{\def\PY@tc##1{\textcolor[rgb]{0.00,0.63,0.00}{##1}}}
\expandafter\def\csname PY@tok@no\endcsname{\def\PY@tc##1{\textcolor[rgb]{0.53,0.00,0.00}{##1}}}
\expandafter\def\csname PY@tok@c1\endcsname{\let\PY@it=\textit\def\PY@tc##1{\textcolor[rgb]{0.25,0.50,0.50}{##1}}}

\def\PYZbs{\char`\\}
\def\PYZus{\char`\_}
\def\PYZob{\char`\{}
\def\PYZcb{\char`\}}
\def\PYZca{\char`\^}
\def\PYZam{\char`\&}
\def\PYZlt{\char`\<}
\def\PYZgt{\char`\>}
\def\PYZsh{\char`\#}
\def\PYZpc{\char`\%}
\def\PYZdl{\char`\$}
\def\PYZhy{\char`\-}
\def\PYZsq{\char`\'}
\def\PYZdq{\char`\"}
\def\PYZti{\char`\~}
% for compatibility with earlier versions
\def\PYZat{@}
\def\PYZlb{[}
\def\PYZrb{]}
\makeatother


    % Exact colors from NB
    \definecolor{incolor}{rgb}{0.0, 0.0, 0.5}
    \definecolor{outcolor}{rgb}{0.545, 0.0, 0.0}



    
    % Prevent overflowing lines due to hard-to-break entities
    \sloppy 
    % Setup hyperref package
    \hypersetup{
      breaklinks=true,  % so long urls are correctly broken across lines
      colorlinks=true,
      urlcolor=urlcolor,
      linkcolor=linkcolor,
      citecolor=citecolor,
      }
    % Slightly bigger margins than the latex defaults
    
    \geometry{verbose,tmargin=1in,bmargin=1in,lmargin=1in,rmargin=1in}
    
    

    \begin{document}
    
    
    %\maketitle
    \setcounter{section}{1}
    

    
    \section{Feature selection}\label{feature-selection}

    Pro 2 dimenze umíme chování funkce vizualizovat dobře, viz např. matice
hexbinplotů v úvodu. Techniky \emph{výběru rysů} (angl. feature
selection) nám umožní vybrat dvě dimenze, které jsou z nějakého pohledu
ty "nejzajímavější". Narozdíl od obecnějších technik \emph{extrakce
rysů} (angl. feature extraction), budou vybrané dimenze vždy již mezi
těmi vstupními.

Pojem "nejzajímavější dimenze" může znamenat ledacos, např.
nejinformativnější vzhledem k vlivu na funkční hodnotu. Ale i
nejinformativnější může znamenat ledacos. To uvidíme u jednotlivých
technik. Obecný postup spočívá v natrénování nějakého statistkého
regresoru a zvolení dvou dimenzí, kterým byla přiřazena největší váha.

    \subsection{Příprava notebooku}\label{pux159uxedprava-notebooku}

    \begin{Verbatim}[commandchars=\\\{\}]
{\color{incolor}In [{\color{incolor}117}]:} \PY{o}{\PYZpc{}}\PY{k}{matplotlib} inline
\end{Verbatim}

    \begin{Verbatim}[commandchars=\\\{\}]
{\color{incolor}In [{\color{incolor}118}]:} \PY{k+kn}{import} \PY{n+nn}{ipywidgets} \PY{k}{as} \PY{n+nn}{widgets}
          \PY{k+kn}{import} \PY{n+nn}{matplotlib}\PY{n+nn}{.}\PY{n+nn}{pyplot} \PY{k}{as} \PY{n+nn}{plt}
          \PY{k+kn}{import} \PY{n+nn}{numpy} \PY{k}{as} \PY{n+nn}{np}
          \PY{k+kn}{import} \PY{n+nn}{pandas} \PY{k}{as} \PY{n+nn}{pd}
          \PY{k+kn}{import} \PY{n+nn}{seaborn} \PY{k}{as} \PY{n+nn}{sns}
          \PY{k+kn}{from} \PY{n+nn}{sklearn}\PY{n+nn}{.}\PY{n+nn}{feature\PYZus{}selection} \PY{k}{import} \PY{n}{SelectKBest}
          \PY{k+kn}{from} \PY{n+nn}{sklearn}\PY{n+nn}{.}\PY{n+nn}{feature\PYZus{}selection} \PY{k}{import} \PY{n}{f\PYZus{}regression}\PY{p}{,} \PY{n}{mutual\PYZus{}info\PYZus{}regression}
          \PY{k+kn}{from} \PY{n+nn}{sklearn}\PY{n+nn}{.}\PY{n+nn}{tree} \PY{k}{import} \PY{n}{DecisionTreeRegressor}
          \PY{k+kn}{from} \PY{n+nn}{sklearn}\PY{n+nn}{.}\PY{n+nn}{ensemble} \PY{k}{import} \PY{n}{RandomForestRegressor}
\end{Verbatim}

    \begin{Verbatim}[commandchars=\\\{\}]
{\color{incolor}In [{\color{incolor}119}]:} \PY{n}{sns}\PY{o}{.}\PY{n}{set}\PY{p}{(}\PY{n}{style}\PY{o}{=}\PY{l+s+s1}{\PYZsq{}}\PY{l+s+s1}{white}\PY{l+s+s1}{\PYZsq{}}\PY{p}{)}
\end{Verbatim}

    \subsection{Načtení dat}\label{naux10dtenuxed-dat}

    \begin{Verbatim}[commandchars=\\\{\}]
{\color{incolor}In [{\color{incolor}120}]:} \PY{n}{d1} \PY{o}{=} \PY{n}{pd}\PY{o}{.}\PY{n}{read\PYZus{}csv}\PY{p}{(}\PY{l+s+s1}{\PYZsq{}}\PY{l+s+s1}{data/data1.csv}\PY{l+s+s1}{\PYZsq{}}\PY{p}{,} \PY{n}{sep}\PY{o}{=}\PY{l+s+s1}{\PYZsq{}}\PY{l+s+s1}{;}\PY{l+s+s1}{\PYZsq{}}\PY{p}{,} \PY{n}{header}\PY{o}{=}\PY{k+kc}{None}\PY{p}{)}
          \PY{n}{n} \PY{o}{=} \PY{l+m+mi}{13}  \PY{c+c1}{\PYZsh{} number of inputs}
          \PY{n}{input\PYZus{}indices} \PY{o}{=} \PY{n+nb}{list}\PY{p}{(}\PY{n+nb}{range}\PY{p}{(}\PY{n}{n}\PY{p}{)}\PY{p}{)}
          \PY{n}{x} \PY{o}{=} \PY{n}{d1}\PY{p}{[}\PY{n}{input\PYZus{}indices}\PY{p}{]}
          \PY{n}{y} \PY{o}{=} \PY{n}{d1}\PY{p}{[}\PY{n}{n}\PY{p}{]}
\end{Verbatim}

    \subsection{Korelace}\label{korelace}

    Nejprve se podívme na to, jak moc korelují jednotlivé vstupní dimenze s
výstupní hodnotou.

    \begin{Verbatim}[commandchars=\\\{\}]
{\color{incolor}In [{\color{incolor}121}]:} \PY{n}{d1}\PY{o}{.}\PY{n}{corr}\PY{p}{(}\PY{p}{)}\PY{o}{.}\PY{n}{abs}\PY{p}{(}\PY{p}{)}\PY{p}{[}\PY{l+m+mi}{13}\PY{p}{]}\PY{p}{[}\PY{n}{input\PYZus{}indices}\PY{p}{]}\PY{o}{.}\PY{n}{plot}\PY{p}{(}\PY{n}{kind}\PY{o}{=}\PY{l+s+s1}{\PYZsq{}}\PY{l+s+s1}{bar}\PY{l+s+s1}{\PYZsq{}}\PY{p}{)}
\end{Verbatim}

            \begin{Verbatim}[commandchars=\\\{\}]
{\color{outcolor}Out[{\color{outcolor}121}]:} <matplotlib.axes.\_subplots.AxesSubplot at 0x7f12a774a978>
\end{Verbatim}
        
    \begin{center}
    \adjustimage{max size={0.9\linewidth}{0.9\paperheight}}{feature-selection_files/feature-selection_10_1.png}
    \end{center}
    { \hspace*{\fill} \\}
    
    Vidíme, že s funkční hodnotou nejvíce korelují dimenze 1 a 10. Můžeme si
teď vykreslit všechny body zprojektované do dimenzí 1 a 10. Každý bod
obarvíme podle jeho funkční hodnoty.

    \begin{Verbatim}[commandchars=\\\{\}]
{\color{incolor}In [{\color{incolor}122}]:} \PY{n}{d1}\PY{o}{.}\PY{n}{plot}\PY{o}{.}\PY{n}{scatter}\PY{p}{(}\PY{n}{x}\PY{o}{=}\PY{l+m+mi}{10}\PY{p}{,} \PY{n}{y}\PY{o}{=}\PY{l+m+mi}{1}\PY{p}{,} \PY{n}{c}\PY{o}{=}\PY{n}{n}\PY{p}{,} \PY{n}{cmap}\PY{o}{=}\PY{l+s+s1}{\PYZsq{}}\PY{l+s+s1}{viridis}\PY{l+s+s1}{\PYZsq{}}\PY{p}{)}
\end{Verbatim}

            \begin{Verbatim}[commandchars=\\\{\}]
{\color{outcolor}Out[{\color{outcolor}122}]:} <matplotlib.axes.\_subplots.AxesSubplot at 0x7f12a7e89828>
\end{Verbatim}
        
    \begin{center}
    \adjustimage{max size={0.9\linewidth}{0.9\paperheight}}{feature-selection_files/feature-selection_12_1.png}
    \end{center}
    { \hspace*{\fill} \\}
    
    Tím, že jsme velký počet dimenzí zredukovali na pouhé dvě, dostalo se
spousta bodů přes sebe a tak v tomto grafu není nic moc vidět. Všechny
vysoké hodnoty jsou zakryté hromadou těch nízkých. Řešením je
hexbinplot, který pro každý úsek dvojdimenzionální plochy zobrazí
maximální dosažené hodnoty (případně průměrnou hodnotu).

    \begin{Verbatim}[commandchars=\\\{\}]
{\color{incolor}In [{\color{incolor}123}]:} \PY{k}{def} \PY{n+nf}{interactive\PYZus{}hexbin\PYZus{}plot}\PY{p}{(}\PY{n}{df}\PY{p}{,} \PY{n}{x}\PY{p}{,} \PY{n}{y}\PY{p}{,} \PY{n}{C}\PY{o}{=}\PY{n}{n}\PY{p}{)}\PY{p}{:}
              \PY{k}{def} \PY{n+nf}{plot\PYZus{}hexbin}\PY{p}{(}\PY{n}{gridsize}\PY{p}{,} \PY{n}{logscale}\PY{p}{,} \PY{n}{maximum}\PY{p}{)}\PY{p}{:}
                  \PY{n}{df}\PY{o}{.}\PY{n}{plot}\PY{o}{.}\PY{n}{hexbin}\PY{p}{(}\PY{n}{x}\PY{o}{=}\PY{n}{x}\PY{p}{,} \PY{n}{y}\PY{o}{=}\PY{n}{y}\PY{p}{,} \PY{n}{C}\PY{o}{=}\PY{n}{C}\PY{p}{,}
                                \PY{n}{bins}\PY{o}{=}\PY{l+s+s1}{\PYZsq{}}\PY{l+s+s1}{log}\PY{l+s+s1}{\PYZsq{}} \PY{k}{if} \PY{n}{logscale} \PY{k}{else} \PY{k+kc}{None}\PY{p}{,}
                                \PY{n}{gridsize}\PY{o}{=}\PY{n}{gridsize}\PY{p}{,}
                                \PY{n}{reduce\PYZus{}C\PYZus{}function}\PY{o}{=}\PY{n}{np}\PY{o}{.}\PY{n}{max} \PY{k}{if} \PY{n}{maximum} \PY{k}{else} \PY{n}{np}\PY{o}{.}\PY{n}{mean}\PY{p}{,}
                                \PY{n}{cmap}\PY{o}{=}\PY{l+s+s1}{\PYZsq{}}\PY{l+s+s1}{viridis}\PY{l+s+s1}{\PYZsq{}}\PY{p}{)}
          
              \PY{n}{gridsize} \PY{o}{=} \PY{n}{widgets}\PY{o}{.}\PY{n}{IntSlider}\PY{p}{(}\PY{n+nb}{min}\PY{o}{=}\PY{l+m+mi}{10}\PY{p}{,} \PY{n+nb}{max}\PY{o}{=}\PY{l+m+mi}{100}\PY{p}{,} \PY{n}{step}\PY{o}{=}\PY{l+m+mi}{5}\PY{p}{,} \PY{n}{value}\PY{o}{=}\PY{l+m+mi}{25}\PY{p}{,} \PY{n}{description}\PY{o}{=}\PY{l+s+s1}{\PYZsq{}}\PY{l+s+s1}{grid size}\PY{l+s+s1}{\PYZsq{}}\PY{p}{)}
              \PY{n}{widgets}\PY{o}{.}\PY{n}{interact}\PY{p}{(}\PY{n}{plot\PYZus{}hexbin}\PY{p}{,} \PY{n}{gridsize}\PY{o}{=}\PY{n}{gridsize}\PY{p}{,} \PY{n}{logscale}\PY{o}{=}\PY{k+kc}{True}\PY{p}{,} \PY{n}{maximum}\PY{o}{=}\PY{k+kc}{True}\PY{p}{)}
\end{Verbatim}

    \begin{Verbatim}[commandchars=\\\{\}]
{\color{incolor}In [{\color{incolor}124}]:} \PY{n}{interactive\PYZus{}hexbin\PYZus{}plot}\PY{p}{(}\PY{n}{d1}\PY{p}{,} \PY{n}{x}\PY{o}{=}\PY{l+m+mi}{10}\PY{p}{,} \PY{n}{y}\PY{o}{=}\PY{l+m+mi}{1}\PY{p}{)}
\end{Verbatim}

    \begin{center}
    \adjustimage{max size={0.9\linewidth}{0.9\paperheight}}{feature-selection_files/feature-selection_15_0.png}
    \end{center}
    { \hspace*{\fill} \\}
    
    Korelace ale popisuje lineární závislost a závislost mezi vstupními
dimenzemi a funkční hodnotou může být zcela nelineární. Což je i náš
případ, na grafu korelací výše vidíme, že jednotlivé korelace jsou
mizivé (\textless{} 0.01).

V dalších částech prozkoumáme další míry "zajímavosti", které lze místo
korelace použít: F-hodnotu, \emph{mutual-information} a
\emph{Gini-index}.

    \subsection{F-hodnota}\label{f-hodnota}

    F-hodnota odpovídá váze natrénovaného lineárního regresoru a jde tak
pouze o upravenou korelaci mezi každou vstupní dimenzí a funkční
hodnotou. P-hodnota pak popisuje pravděpodobnost pozorované korelace,
pokud by skutečná korelace byla nulová. Takže dimenze je pro lineární
regresi tím důležitější, čím je F-hodnota větší a čím je p-hodnota
menší. Podívejme se na graf F-hodnot a p-hodnot:

    \begin{Verbatim}[commandchars=\\\{\}]
{\color{incolor}In [{\color{incolor}125}]:} \PY{n}{dimensions\PYZus{}selector} \PY{o}{=} \PY{n}{SelectKBest}\PY{p}{(}\PY{n}{f\PYZus{}regression}\PY{p}{,} \PY{n}{k}\PY{o}{=}\PY{l+m+mi}{2}\PY{p}{)}
          \PY{n}{dimensions\PYZus{}selector}\PY{o}{.}\PY{n}{fit}\PY{p}{(}\PY{n}{x}\PY{p}{,} \PY{n}{y}\PY{p}{)}
          \PY{n}{pvalues} \PY{o}{=} \PY{n}{pd}\PY{o}{.}\PY{n}{Series}\PY{p}{(}\PY{n}{dimensions\PYZus{}selector}\PY{o}{.}\PY{n}{pvalues\PYZus{}}\PY{p}{,} \PY{n}{name}\PY{o}{=}\PY{l+s+s1}{\PYZsq{}}\PY{l+s+s1}{p\PYZhy{}value}\PY{l+s+s1}{\PYZsq{}}\PY{p}{)}
          \PY{n}{scores} \PY{o}{=} \PY{n}{pd}\PY{o}{.}\PY{n}{Series}\PY{p}{(}\PY{n}{dimensions\PYZus{}selector}\PY{o}{.}\PY{n}{scores\PYZus{}}\PY{p}{,} \PY{n}{name}\PY{o}{=}\PY{l+s+s1}{\PYZsq{}}\PY{l+s+s1}{score}\PY{l+s+s1}{\PYZsq{}}\PY{p}{)}
          \PY{n}{dim\PYZus{}info} \PY{o}{=} \PY{n}{pd}\PY{o}{.}\PY{n}{concat}\PY{p}{(}\PY{p}{[}\PY{n}{pvalues}\PY{p}{,} \PY{n}{scores}\PY{p}{]}\PY{p}{,} \PY{n}{axis}\PY{o}{=}\PY{l+m+mi}{1}\PY{p}{)}
          \PY{n}{dim\PYZus{}info}\PY{o}{.}\PY{n}{plot}\PY{p}{(}\PY{n}{kind}\PY{o}{=}\PY{l+s+s1}{\PYZsq{}}\PY{l+s+s1}{bar}\PY{l+s+s1}{\PYZsq{}}\PY{p}{)}
\end{Verbatim}

            \begin{Verbatim}[commandchars=\\\{\}]
{\color{outcolor}Out[{\color{outcolor}125}]:} <matplotlib.axes.\_subplots.AxesSubplot at 0x7f12aa3b52b0>
\end{Verbatim}
        
    \begin{center}
    \adjustimage{max size={0.9\linewidth}{0.9\paperheight}}{feature-selection_files/feature-selection_19_1.png}
    \end{center}
    { \hspace*{\fill} \\}
    
    Největší skóre (a nejmenší p-hodnotu) mají dimenze 10 a 1. To je stejný
výsledek jako u korelace, což není překvapivé, protože F-hodnota se
počítá z korelace.

    \begin{Verbatim}[commandchars=\\\{\}]
{\color{incolor}In [{\color{incolor}126}]:} \PY{n}{interactive\PYZus{}hexbin\PYZus{}plot}\PY{p}{(}\PY{n}{d1}\PY{p}{,} \PY{n}{x}\PY{o}{=}\PY{l+m+mi}{10}\PY{p}{,} \PY{n}{y}\PY{o}{=}\PY{l+m+mi}{1}\PY{p}{)}
\end{Verbatim}

    \begin{center}
    \adjustimage{max size={0.9\linewidth}{0.9\paperheight}}{feature-selection_files/feature-selection_21_0.png}
    \end{center}
    { \hspace*{\fill} \\}
    
    \subsection{Vzájemná informace}\label{vzuxe1jemnuxe1-informace}

    Vzájemná informace (angl.
\href{https://en.wikipedia.org/wiki/Mutual_information}{mutual-information})
je metrika vyjadřující, jak znalost jedné proměnné snižuje nejistotu
ohledně hodnoty druhé proměnné. Základní výhodou oproti korelaci a
F-hodnotě je, že umí zachytit libovolnou závislost, nikoliv jen tu
lineární. Nevýhodou je složitější, časově náročnější výpočet.

Vypočtěme si tedy vzájemnou informaci mezi každou ze vstupních
proměnných a funkční hodnotou:

    \begin{Verbatim}[commandchars=\\\{\}]
{\color{incolor}In [{\color{incolor}127}]:} \PY{n}{dimensions\PYZus{}selector\PYZus{}mi} \PY{o}{=} \PY{n}{SelectKBest}\PY{p}{(}\PY{n}{mutual\PYZus{}info\PYZus{}regression}\PY{p}{,} \PY{n}{k}\PY{o}{=}\PY{l+m+mi}{2}\PY{p}{)}
          \PY{n}{ds} \PY{o}{=} \PY{n}{d1}\PY{o}{.}\PY{n}{sample}\PY{p}{(}\PY{l+m+mi}{10000}\PY{p}{,} \PY{n}{random\PYZus{}state}\PY{o}{=}\PY{l+m+mi}{0}\PY{p}{)}
          \PY{n}{xs} \PY{o}{=} \PY{n}{ds}\PY{p}{[}\PY{n}{input\PYZus{}indices}\PY{p}{]}
          \PY{n}{ys} \PY{o}{=} \PY{n}{ds}\PY{p}{[}\PY{n}{n}\PY{p}{]}
          \PY{n}{dimensions\PYZus{}selector\PYZus{}mi}\PY{o}{.}\PY{n}{fit}\PY{p}{(}\PY{n}{xs}\PY{p}{,} \PY{n}{ys}\PY{p}{)}
          \PY{n}{scores\PYZus{}mi} \PY{o}{=} \PY{n}{pd}\PY{o}{.}\PY{n}{Series}\PY{p}{(}\PY{n}{dimensions\PYZus{}selector\PYZus{}mi}\PY{o}{.}\PY{n}{scores\PYZus{}}\PY{p}{,} \PY{n}{name}\PY{o}{=}\PY{l+s+s1}{\PYZsq{}}\PY{l+s+s1}{score}\PY{l+s+s1}{\PYZsq{}}\PY{p}{)}
          \PY{n+nb}{print}\PY{p}{(}\PY{n}{scores\PYZus{}mi}\PY{o}{.}\PY{n}{sort\PYZus{}values}\PY{p}{(}\PY{n}{ascending}\PY{o}{=}\PY{k+kc}{False}\PY{p}{)}\PY{p}{)}
          \PY{n}{scores\PYZus{}mi}\PY{o}{.}\PY{n}{plot}\PY{p}{(}\PY{n}{kind}\PY{o}{=}\PY{l+s+s1}{\PYZsq{}}\PY{l+s+s1}{bar}\PY{l+s+s1}{\PYZsq{}}\PY{p}{)}
\end{Verbatim}

    \begin{Verbatim}[commandchars=\\\{\}]
0     3.404539
10    2.964799
5     2.928335
9     2.902886
1     2.896619
6     2.872758
3     2.858247
8     2.820112
11    2.803946
12    2.779271
4     2.773454
7     2.752495
2     2.737586
Name: score, dtype: float64

    \end{Verbatim}

            \begin{Verbatim}[commandchars=\\\{\}]
{\color{outcolor}Out[{\color{outcolor}127}]:} <matplotlib.axes.\_subplots.AxesSubplot at 0x7f12bb2846d8>
\end{Verbatim}
        
    \begin{center}
    \adjustimage{max size={0.9\linewidth}{0.9\paperheight}}{feature-selection_files/feature-selection_24_2.png}
    \end{center}
    { \hspace*{\fill} \\}
    
    Největší skóre teď mají dimenze 0 a 10. Projekce funkce do prostoru
daného těmito dvěma dimenzemi vypadá následovně:

    \begin{Verbatim}[commandchars=\\\{\}]
{\color{incolor}In [{\color{incolor}128}]:} \PY{n}{interactive\PYZus{}hexbin\PYZus{}plot}\PY{p}{(}\PY{n}{d1}\PY{p}{,} \PY{n}{x}\PY{o}{=}\PY{l+m+mi}{0}\PY{p}{,} \PY{n}{y}\PY{o}{=}\PY{l+m+mi}{10}\PY{p}{)}
\end{Verbatim}

    \begin{center}
    \adjustimage{max size={0.9\linewidth}{0.9\paperheight}}{feature-selection_files/feature-selection_26_0.png}
    \end{center}
    { \hspace*{\fill} \\}
    
    \subsection{Rozhodovací stromy}\label{rozhodovacuxed-stromy}

    Dalším modelem, který lze použít pro výběr rysů jsou regresní
rozhodovací stromy (angl. decision trees). Naučený regresor opět
poskytuje informaci o důležitosti rysů, podle toho, jak moc která
vstupní dimenze přispívá k celkovému rozhodnutí o predikované hodnotě.
Této metrice se říká \emph{Gini importance}.

    \begin{Verbatim}[commandchars=\\\{\}]
{\color{incolor}In [{\color{incolor}129}]:} \PY{n}{tree\PYZus{}model} \PY{o}{=} \PY{n}{DecisionTreeRegressor}\PY{p}{(}\PY{p}{)}
          \PY{n}{tree\PYZus{}model} \PY{o}{=} \PY{n}{tree\PYZus{}model}\PY{o}{.}\PY{n}{fit}\PY{p}{(}\PY{n}{x}\PY{p}{,} \PY{n}{y}\PY{p}{)}
          \PY{n}{pd}\PY{o}{.}\PY{n}{Series}\PY{p}{(}\PY{n}{tree\PYZus{}model}\PY{o}{.}\PY{n}{feature\PYZus{}importances\PYZus{}}\PY{p}{)}\PY{o}{.}\PY{n}{plot}\PY{p}{(}\PY{n}{kind}\PY{o}{=}\PY{l+s+s1}{\PYZsq{}}\PY{l+s+s1}{bar}\PY{l+s+s1}{\PYZsq{}}\PY{p}{,} \PY{n}{logy}\PY{o}{=}\PY{k+kc}{True}\PY{p}{)}
\end{Verbatim}

            \begin{Verbatim}[commandchars=\\\{\}]
{\color{outcolor}Out[{\color{outcolor}129}]:} <matplotlib.axes.\_subplots.AxesSubplot at 0x7f12b347cc18>
\end{Verbatim}
        
    \begin{center}
    \adjustimage{max size={0.9\linewidth}{0.9\paperheight}}{feature-selection_files/feature-selection_29_1.png}
    \end{center}
    { \hspace*{\fill} \\}
    
    Problém rozhodovacího stromu je ten, že není stabilní, každý běh může
skončit se značně jiným stromem a tedy značně jinými důležitostmi
dimenzí. Řešením je les stromů!

    \subsection{Náhodný les}\label{nuxe1hodnuxfd-les}

    Náhodný les (angl. \emph{random forest}) je model, který průměruje
hodnoty z mnoha rozhodovacích stromů, při jejichž učení bylo využito
náhodnosti. Náhodný les je vhodnější pro výběr rysů než rozhdovací
strom, protože díky průměrování velkého množství stromů je vypočtená
důležitost jednotlivých rysů robustnější (má menší rozptyl).

    \begin{Verbatim}[commandchars=\\\{\}]
{\color{incolor}In [{\color{incolor}130}]:} \PY{n}{forest\PYZus{}model} \PY{o}{=} \PY{n}{RandomForestRegressor}\PY{p}{(}\PY{n}{random\PYZus{}state}\PY{o}{=}\PY{l+m+mi}{0}\PY{p}{)}
          \PY{n}{forest\PYZus{}model} \PY{o}{=} \PY{n}{forest\PYZus{}model}\PY{o}{.}\PY{n}{fit}\PY{p}{(}\PY{n}{x}\PY{p}{,} \PY{n}{y}\PY{p}{)}
          \PY{n}{pd}\PY{o}{.}\PY{n}{Series}\PY{p}{(}\PY{n}{forest\PYZus{}model}\PY{o}{.}\PY{n}{feature\PYZus{}importances\PYZus{}}\PY{p}{)}\PY{o}{.}\PY{n}{plot}\PY{p}{(}\PY{n}{kind}\PY{o}{=}\PY{l+s+s1}{\PYZsq{}}\PY{l+s+s1}{bar}\PY{l+s+s1}{\PYZsq{}}\PY{p}{,} \PY{n}{logy}\PY{o}{=}\PY{k+kc}{True}\PY{p}{)}
\end{Verbatim}

            \begin{Verbatim}[commandchars=\\\{\}]
{\color{outcolor}Out[{\color{outcolor}130}]:} <matplotlib.axes.\_subplots.AxesSubplot at 0x7f12a8f44470>
\end{Verbatim}
        
    \begin{center}
    \adjustimage{max size={0.9\linewidth}{0.9\paperheight}}{feature-selection_files/feature-selection_33_1.png}
    \end{center}
    { \hspace*{\fill} \\}
    
    Nejvýznamnější dimenze: 11 a 2

    \begin{Verbatim}[commandchars=\\\{\}]
{\color{incolor}In [{\color{incolor}131}]:} \PY{n}{interactive\PYZus{}hexbin\PYZus{}plot}\PY{p}{(}\PY{n}{d1}\PY{p}{,} \PY{n}{x}\PY{o}{=}\PY{l+m+mi}{11}\PY{p}{,} \PY{n}{y}\PY{o}{=}\PY{l+m+mi}{2}\PY{p}{)}
\end{Verbatim}

    \begin{center}
    \adjustimage{max size={0.9\linewidth}{0.9\paperheight}}{feature-selection_files/feature-selection_35_0.png}
    \end{center}
    { \hspace*{\fill} \\}
    
    \subsection{Výsledky}\label{vuxfdsledky}

    Přehled nejvýznamnějších dimenzí podle různých metrik:

\begin{longtable}[]{@{}lll@{}}
\toprule
Metrika (model) & d1 & d2\tabularnewline
\midrule
\endhead
Korelace / F-hodnota (lineární regrese) & 10 & 1\tabularnewline
Vzájemná informace & 0 & 10\tabularnewline
Gini importance (Náhodný les) & 11 & 2\tabularnewline
\bottomrule
\end{longtable}

    Provedené analýzy můžeme také využít tak, že zjednodušíme graf z úvodu,
který zobrazoval projekci na všechny dvojice vstupních hodnot. Omezíme
se jen na několik dimenzí, které se podle analýz zdají být
nejzajímavější. Navíc budeme zobrazovat zlogaritmované funkční hodnoty.

    \begin{Verbatim}[commandchars=\\\{\}]
{\color{incolor}In [{\color{incolor}132}]:} \PY{n}{ds} \PY{o}{=} \PY{n}{d1}\PY{o}{.}\PY{n}{sample}\PY{p}{(}\PY{l+m+mi}{10000}\PY{p}{,} \PY{n}{random\PYZus{}state}\PY{o}{=}\PY{l+m+mi}{0}\PY{p}{)}
          \PY{n}{y\PYZus{}log} \PY{o}{=} \PY{n}{ds}\PY{p}{[}\PY{n}{n}\PY{p}{]}\PY{o}{.}\PY{n}{apply}\PY{p}{(}\PY{n}{np}\PY{o}{.}\PY{n}{log}\PY{p}{)}
          \PY{n}{g} \PY{o}{=} \PY{n}{sns}\PY{o}{.}\PY{n}{PairGrid}\PY{p}{(}\PY{n}{ds}\PY{p}{,} \PY{n+nb}{vars}\PY{o}{=}\PY{p}{[}\PY{l+m+mi}{0}\PY{p}{,} \PY{l+m+mi}{2}\PY{p}{,} \PY{l+m+mi}{10}\PY{p}{,} \PY{l+m+mi}{11}\PY{p}{]}\PY{p}{)}
          \PY{n}{g} \PY{o}{=} \PY{n}{g}\PY{o}{.}\PY{n}{map\PYZus{}diag}\PY{p}{(}\PY{n}{plt}\PY{o}{.}\PY{n}{scatter}\PY{p}{,} \PY{n}{y}\PY{o}{=}\PY{n}{y\PYZus{}log}\PY{p}{)}
          \PY{n}{g} \PY{o}{=} \PY{n}{g}\PY{o}{.}\PY{n}{map\PYZus{}offdiag}\PY{p}{(}\PY{n}{plt}\PY{o}{.}\PY{n}{hexbin}\PY{p}{,} \PY{n}{C}\PY{o}{=}\PY{n}{y\PYZus{}log}\PY{p}{,} \PY{n}{reduce\PYZus{}C\PYZus{}function}\PY{o}{=}\PY{n}{np}\PY{o}{.}\PY{n}{max}\PY{p}{,} \PY{n}{gridsize}\PY{o}{=}\PY{l+m+mi}{15}\PY{p}{,} \PY{n}{cmap}\PY{o}{=}\PY{l+s+s1}{\PYZsq{}}\PY{l+s+s1}{viridis}\PY{l+s+s1}{\PYZsq{}}\PY{p}{)}
\end{Verbatim}

    \begin{center}
    \adjustimage{max size={0.9\linewidth}{0.9\paperheight}}{feature-selection_files/feature-selection_39_0.png}
    \end{center}
    { \hspace*{\fill} \\}
    
    \subsection{Větší dataset}\label{vux11btux161uxed-dataset}

    Zkusme pomocí uvedených technik výběru rysů prozkoumat ještě druhý
dataset. Protože je mnohonásobně větší, některé výpočty by trvaly
dlouho. To lze vyřešit náhodným nasemplováním části bodů.

    \begin{Verbatim}[commandchars=\\\{\}]
{\color{incolor}In [{\color{incolor}133}]:} \PY{n}{d2} \PY{o}{=} \PY{n}{pd}\PY{o}{.}\PY{n}{read\PYZus{}csv}\PY{p}{(}\PY{l+s+s1}{\PYZsq{}}\PY{l+s+s1}{data/data2.csv}\PY{l+s+s1}{\PYZsq{}}\PY{p}{,} \PY{n}{sep}\PY{o}{=}\PY{l+s+s1}{\PYZsq{}}\PY{l+s+s1}{;}\PY{l+s+s1}{\PYZsq{}}\PY{p}{,} \PY{n}{header}\PY{o}{=}\PY{k+kc}{None}\PY{p}{)}
          \PY{n}{x2}\PY{p}{,} \PY{n}{y2} \PY{o}{=} \PY{n}{d2}\PY{p}{[}\PY{n}{input\PYZus{}indices}\PY{p}{]}\PY{p}{,} \PY{n}{d2}\PY{p}{[}\PY{n}{n}\PY{p}{]}
          \PY{n}{ds2} \PY{o}{=} \PY{n}{d2}\PY{o}{.}\PY{n}{sample}\PY{p}{(}\PY{l+m+mi}{10000}\PY{p}{,} \PY{n}{random\PYZus{}state}\PY{o}{=}\PY{l+m+mi}{0}\PY{p}{)}
          \PY{n}{xs2}\PY{p}{,} \PY{n}{ys2} \PY{o}{=} \PY{n}{ds2}\PY{p}{[}\PY{n}{input\PYZus{}indices}\PY{p}{]}\PY{p}{,} \PY{n}{ds2}\PY{p}{[}\PY{n}{n}\PY{p}{]}
\end{Verbatim}

    \begin{Verbatim}[commandchars=\\\{\}]
{\color{incolor}In [{\color{incolor}134}]:} \PY{n}{plt}\PY{o}{.}\PY{n}{title}\PY{p}{(}\PY{l+s+s1}{\PYZsq{}}\PY{l+s+s1}{F\PYZhy{}skóre vstupních dimenzí}\PY{l+s+s1}{\PYZsq{}}\PY{p}{)}
          \PY{n}{dimensions\PYZus{}selector}\PY{o}{.}\PY{n}{fit}\PY{p}{(}\PY{n}{x2}\PY{p}{,} \PY{n}{y2}\PY{p}{)}
          \PY{n}{pd}\PY{o}{.}\PY{n}{Series}\PY{p}{(}\PY{n}{dimensions\PYZus{}selector}\PY{o}{.}\PY{n}{scores\PYZus{}}\PY{p}{,} \PY{n}{name}\PY{o}{=}\PY{l+s+s1}{\PYZsq{}}\PY{l+s+s1}{score}\PY{l+s+s1}{\PYZsq{}}\PY{p}{)}\PY{o}{.}\PY{n}{plot}\PY{p}{(}\PY{n}{kind}\PY{o}{=}\PY{l+s+s1}{\PYZsq{}}\PY{l+s+s1}{bar}\PY{l+s+s1}{\PYZsq{}}\PY{p}{)}
\end{Verbatim}

            \begin{Verbatim}[commandchars=\\\{\}]
{\color{outcolor}Out[{\color{outcolor}134}]:} <matplotlib.axes.\_subplots.AxesSubplot at 0x7f12adc0ddd8>
\end{Verbatim}
        
    \begin{center}
    \adjustimage{max size={0.9\linewidth}{0.9\paperheight}}{feature-selection_files/feature-selection_43_1.png}
    \end{center}
    { \hspace*{\fill} \\}
    
    \begin{Verbatim}[commandchars=\\\{\}]
{\color{incolor}In [{\color{incolor}135}]:} \PY{n}{plt}\PY{o}{.}\PY{n}{title}\PY{p}{(}\PY{l+s+s1}{\PYZsq{}}\PY{l+s+s1}{Vzájemná informace}\PY{l+s+s1}{\PYZsq{}}\PY{p}{)}
          \PY{n}{dimensions\PYZus{}selector\PYZus{}mi}\PY{o}{.}\PY{n}{fit}\PY{p}{(}\PY{n}{xs2}\PY{p}{,} \PY{n}{ys2}\PY{p}{)}
          \PY{n}{scores\PYZus{}mi} \PY{o}{=} \PY{n}{pd}\PY{o}{.}\PY{n}{Series}\PY{p}{(}\PY{n}{dimensions\PYZus{}selector\PYZus{}mi}\PY{o}{.}\PY{n}{scores\PYZus{}}\PY{p}{,} \PY{n}{name}\PY{o}{=}\PY{l+s+s1}{\PYZsq{}}\PY{l+s+s1}{score}\PY{l+s+s1}{\PYZsq{}}\PY{p}{)}
          \PY{n}{scores\PYZus{}mi}\PY{o}{.}\PY{n}{plot}\PY{p}{(}\PY{n}{kind}\PY{o}{=}\PY{l+s+s1}{\PYZsq{}}\PY{l+s+s1}{bar}\PY{l+s+s1}{\PYZsq{}}\PY{p}{)}
\end{Verbatim}

            \begin{Verbatim}[commandchars=\\\{\}]
{\color{outcolor}Out[{\color{outcolor}135}]:} <matplotlib.axes.\_subplots.AxesSubplot at 0x7f12a759bd30>
\end{Verbatim}
        
    \begin{center}
    \adjustimage{max size={0.9\linewidth}{0.9\paperheight}}{feature-selection_files/feature-selection_44_1.png}
    \end{center}
    { \hspace*{\fill} \\}
    
    \begin{Verbatim}[commandchars=\\\{\}]
{\color{incolor}In [{\color{incolor}136}]:} \PY{n}{plt}\PY{o}{.}\PY{n}{title}\PY{p}{(}\PY{l+s+s1}{\PYZsq{}}\PY{l+s+s1}{Gini importance (náhodný les)}\PY{l+s+s1}{\PYZsq{}}\PY{p}{)}
          \PY{n}{forest\PYZus{}model} \PY{o}{=} \PY{n}{forest\PYZus{}model}\PY{o}{.}\PY{n}{fit}\PY{p}{(}\PY{n}{xs2}\PY{p}{,} \PY{n}{ys2}\PY{p}{)}
          \PY{n}{pd}\PY{o}{.}\PY{n}{Series}\PY{p}{(}\PY{n}{forest\PYZus{}model}\PY{o}{.}\PY{n}{feature\PYZus{}importances\PYZus{}}\PY{p}{)}\PY{o}{.}\PY{n}{plot}\PY{p}{(}\PY{n}{kind}\PY{o}{=}\PY{l+s+s1}{\PYZsq{}}\PY{l+s+s1}{bar}\PY{l+s+s1}{\PYZsq{}}\PY{p}{,} \PY{n}{logy}\PY{o}{=}\PY{k+kc}{True}\PY{p}{)}
\end{Verbatim}

            \begin{Verbatim}[commandchars=\\\{\}]
{\color{outcolor}Out[{\color{outcolor}136}]:} <matplotlib.axes.\_subplots.AxesSubplot at 0x7f12aa485198>
\end{Verbatim}
        
    \begin{center}
    \adjustimage{max size={0.9\linewidth}{0.9\paperheight}}{feature-selection_files/feature-selection_45_1.png}
    \end{center}
    { \hspace*{\fill} \\}
    
    Přehled nejvýznamnějších dimenzí podle různých metrik:

\begin{longtable}[]{@{}lll@{}}
\toprule
Metrika (model) & d1 & d2\tabularnewline
\midrule
\endhead
Korelace / F-hodnota (lineární regrese) & 12 & 10\tabularnewline
Vzájemná informace & 0 & 5\tabularnewline
Gini importance (Náhodný les) & 12 & 0\tabularnewline
\bottomrule
\end{longtable}

    Jako dvě nejzajímější dimenze se zdají 0 a 5, vykresleme si tedy nejprve
graf zprojektovaný do těchto dvou dimenzí:

    \begin{Verbatim}[commandchars=\\\{\}]
{\color{incolor}In [{\color{incolor}137}]:} \PY{n}{interactive\PYZus{}hexbin\PYZus{}plot}\PY{p}{(}\PY{n}{ds2}\PY{p}{,} \PY{n}{x}\PY{o}{=}\PY{l+m+mi}{0}\PY{p}{,} \PY{n}{y}\PY{o}{=}\PY{l+m+mi}{5}\PY{p}{)}
\end{Verbatim}

    \begin{center}
    \adjustimage{max size={0.9\linewidth}{0.9\paperheight}}{feature-selection_files/feature-selection_48_0.png}
    \end{center}
    { \hspace*{\fill} \\}
    
    Nebo můžeme vykreslit projekce pro všechny dvojice z těch
nejzajímavějších dimenzí:

    \begin{Verbatim}[commandchars=\\\{\}]
{\color{incolor}In [{\color{incolor}138}]:} \PY{n}{y2\PYZus{}log} \PY{o}{=} \PY{n}{ds2}\PY{p}{[}\PY{n}{n}\PY{p}{]}\PY{o}{.}\PY{n}{apply}\PY{p}{(}\PY{n}{np}\PY{o}{.}\PY{n}{log}\PY{p}{)}
          \PY{n}{g} \PY{o}{=} \PY{n}{sns}\PY{o}{.}\PY{n}{PairGrid}\PY{p}{(}\PY{n}{ds2}\PY{p}{,} \PY{n+nb}{vars}\PY{o}{=}\PY{p}{[}\PY{l+m+mi}{0}\PY{p}{,} \PY{l+m+mi}{5}\PY{p}{,} \PY{l+m+mi}{10}\PY{p}{,} \PY{l+m+mi}{12}\PY{p}{]}\PY{p}{)}
          \PY{n}{g} \PY{o}{=} \PY{n}{g}\PY{o}{.}\PY{n}{map\PYZus{}diag}\PY{p}{(}\PY{n}{plt}\PY{o}{.}\PY{n}{scatter}\PY{p}{,} \PY{n}{y}\PY{o}{=}\PY{n}{y2\PYZus{}log}\PY{p}{)}
          \PY{n}{g} \PY{o}{=} \PY{n}{g}\PY{o}{.}\PY{n}{map\PYZus{}offdiag}\PY{p}{(}\PY{n}{plt}\PY{o}{.}\PY{n}{hexbin}\PY{p}{,} \PY{n}{C}\PY{o}{=}\PY{n}{y2\PYZus{}log}\PY{p}{,} \PY{n}{reduce\PYZus{}C\PYZus{}function}\PY{o}{=}\PY{n}{np}\PY{o}{.}\PY{n}{max}\PY{p}{,} \PY{n}{gridsize}\PY{o}{=}\PY{l+m+mi}{15}\PY{p}{,} \PY{n}{cmap}\PY{o}{=}\PY{l+s+s1}{\PYZsq{}}\PY{l+s+s1}{viridis}\PY{l+s+s1}{\PYZsq{}}\PY{p}{)}
\end{Verbatim}

    \begin{center}
    \adjustimage{max size={0.9\linewidth}{0.9\paperheight}}{feature-selection_files/feature-selection_50_0.png}
    \end{center}
    { \hspace*{\fill} \\}
    

    % Add a bibliography block to the postdoc
    
    
    
    \end{document}
